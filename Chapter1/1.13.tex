\documentclass[a4paper,12pt]{article}
\usepackage[utf8]{inputenc}
\usepackage[T1]{fontenc}
\usepackage{titlesec}
\usepackage{graphicx}
\usepackage{mathtools}
\usepackage{amsmath,amsthm,amssymb,scrextend}
\usepackage[top=1.0in,bottom=1.0in]{geometry}
\usepackage{enumitem}

\usepackage{gensymb}

\begin{document}

\title{SICP\\ Solution to Exercise 1.13}
\author{Felipe Peressim}

\date{}
\maketitle

\textbf{Exercise 1.13:} Prove that $\operatorname{Fib}(n)$ is the closest integer to $\varphi^{n}/\sqrt{5}$, where $\varphi=(1+\sqrt{5})/2$.
Hint: Let $\psi=(1-\sqrt{5})/2$. Use induction and the definition of the Fibonacci numbers (see 1.2.2)
to prove that $\operatorname{Fib}(n)=(\varphi^{n}-\psi^{n})/\sqrt{5}$.\\ \\

\begin{proof}
  We use strong induction. Let $n \in \mathbb{N}$ be arbitrary, and suppose that for all $k < n$,

 
  $$\operatorname{Fib}(k)=\frac{\left(\frac{1+\sqrt{5}}{2}\right)^{k}-\left(\frac{1-\sqrt{5}}{2}\right)^{k}}{\sqrt{5}}$$

  If $n = 0$, then

\begin{align*}
  \frac{\left(\frac{1+\sqrt{5}}{2}\right)^{n}-\left(\frac{1-\sqrt{5}}{2}\right)^{n}}{\sqrt{5}}
  &=\frac{\left(\frac{1+\sqrt{5}}{2}\right)^{0}-\left(\frac{1-\sqrt{5}}{2}\right)^{0}}{\sqrt{5}} \\
  &=\frac{1-1}{\sqrt{5}}=0=\operatorname{Fib}(0).
\end{align*}

If $n = 1$, then

\begin{align*}
  \frac{\left(\frac{1+\sqrt{5}}{2}\right)^{n}-\left(\frac{1-\sqrt{5}}{2}\right)^{n}}{\sqrt{5}}
  &=\frac{\left(\frac{1+\sqrt{5}}{2}\right)^{1}-\left(\frac{1-\sqrt{5}}{2}\right)^{1}}{\sqrt{5}} \\
  &=\frac{\sqrt{5}}{\sqrt{5}}=1=\operatorname{Fib}(1).
\end{align*}


For $n \geq 2$, applying the inductive hypothesis to $n-2$ and $n-1$, we have


\begin{align*}
  \operatorname{Fib}(n) &= \operatorname{Fib}(n-2) + \operatorname{Fib}(n-1)\\
  &=\frac{\left(\frac{1+\sqrt{5}}{2}\right)^{n-2} - \left(\frac{1-\sqrt{5}}{2}\right)^{n-2}}{\sqrt{5}} +
    \frac{\left(\frac{1+\sqrt{5}}{2}\right)^{n-1} - \left(\frac{1-\sqrt{5}}{2}\right)^{n-1}}{\sqrt{5}}\\
\end{align*}
\begin{align*}
  &=\frac{\left(\frac{1+\sqrt{5}}{2}\right)^{n-2}+\left(\frac{1+\sqrt{5}}{2}\right)^{n-1} -
    \left(\frac{1-\sqrt{5}}{2}\right)^{n-2}-\left(\frac{1-\sqrt{5}}{2}\right)^{n-1}}{\sqrt{5}}\\
  &= \frac{\left(\frac{1+\sqrt{5}}{2}\right)^{n-2}\left(1+\frac{1+\sqrt{5}}{2}\right)-\left(\frac{1-\sqrt{5}}{2}\right)^{n-2}\left(1+\frac{1-\sqrt{5}}{2}\right)}{\sqrt{5}}
\end{align*}

Since $\left(\frac{1+\sqrt{5}}{2}\right)^{2}=\frac{3+\sqrt{5}}{2}=1+\frac{1+\sqrt{5}}{2}$

and similarly $\left(\frac{1-\sqrt{5}}{2}\right)^{2}=\frac{3-\sqrt{5}}{2}=1+\frac{1-\sqrt{5}}{2}$.\\

Substituting into the formula for $\operatorname{Fib}(n)$ yields

\begin{align*}
  \operatorname{Fib}(n)&=\frac{\left(\frac{1+\sqrt{5}}{2}\right)^{n-2}\left(\frac{1+\sqrt{5}}{2}\right)^{2}-\left(\frac{1-\sqrt{5}}{2}\right)^{n-2}\left(\frac{1-\sqrt{5}}{2}\right)^{2}}{\sqrt{5}}\\
  &=\frac{\left(\frac{1+\sqrt{5}}{2}\right)^{n}-\left(\frac{1-\sqrt{5}}{2}\right)^{n}}{\sqrt{5}}
\end{align*}

Therefore, Fib($n$)$=(\varphi^{n}-\psi^{n})/\sqrt{5}$ as required.

\end{proof}

Now, we will prove that Fib($n$) is the closest integer to $\varphi^{n}/\sqrt{5}$.\\

\begin{proof}

  Since $\operatorname{Fib}(n)=\frac{(\varphi^{n}-\psi^{n})}{\sqrt{5}}$, it can be rewritten as

  $$\frac{\varphi^{n}}{\sqrt{5}} = \operatorname{Fib}(n) + \frac{\psi^{n}}{\sqrt{5}}$$

  Note that, since $\left|\frac{\psi}{\sqrt{5}}\right| < \frac{1}{2}$, it follows that $\left|\frac{\psi^{n}}{\sqrt{5}}\right| < \left|\frac{\psi}{\sqrt{5}}\right| < \frac{1}{2}$, and
  therefore $-\frac{1}{2} < \frac{\psi^{n}}{\sqrt{5}} < \frac{1}{2}$.\\

  Now adding $\operatorname{Fib}(n)$ to the inequality $-\frac{1}{2} < \frac{\psi^{n}}{\sqrt{5}} < \frac{1}{2}$ gives

  $$\operatorname{Fib}(n) -\frac{1}{2} < \operatorname{Fib}(n) + \frac{\psi^{n}}{\sqrt{5}} < \operatorname{Fib}(n) + \frac{1}{2}$$.

  Since $\frac{\varphi^{n}}{\sqrt{5}} = \operatorname{Fib}(n) + \frac{\psi^{n}}{\sqrt{5}}$, then

  $$\operatorname{Fib}(n) -\frac{1}{2} < \frac{\varphi^{n}}{\sqrt{5}} < \operatorname{Fib}(n) + \frac{1}{2}$$.

  Which shows that $\operatorname{Fib}(n)$ is the closest integer to $\varphi^{n}/\sqrt{5}$.
  
\end{proof}

\end{document}